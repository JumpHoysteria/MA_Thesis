% Some commands used in this file
\renewcommand{\package}{\emph}
\newcommand{\product}{\textit}

\chapter{Material and Methods}

\section{General information}
All the liquids that will be in contact with the cell, are pre-warmed to limit cold-exposure. Unless specifically stated otherwise, we worked under sterile conditions. By courtesy of Dr. Ulrich Blache, we had access to human bone-marrow derived mesenchymal stem cells from anonymized patients with unknown medical history. Incubation parameters are kept constant at 37 deg C and 5\% CO2 in humidified air.

\section{Cell Culture And Passaging}

Cells are grown in growth medium under sterile conditions. Nunc\texttrademark{} EasYFlask\texttrademark{} Cell Culture Flasks with filters (ThermoFisher) were used for growing the cells. The media was changed every three days. Passaging was doneat 80\% confluence using a standard TE-protocol. First, the cells are generously washed with PBS. Second, 2ml Trypsin-EDTA (MANUFACTURER) is added, to detach the cells. After 5-10 minutes cells are round and detached as assessed by light microscopy. If not, a mild horizontal hit of the flask against hard surface can help in detaching the cells. Then, add 10ml MEM\textalpha{ }with 10\% FBS to deactivate the Trypsin-EDTA. Transfer cell solution to 15ml Falcon Tube and centrifuge for 5 minutes at a speed of 500 rcf. Next, discard the supernatant, leaving only the pellet. Resuspend the cells and seed according to application with typical seeding density being 10\textsuperscript{6} Cells / 75 cm\textsuperscript{2}. 
Growth Medium consist of ROTI\textregistered{} Cell Eagle's MEM-Alpha with nucleosides and stable l-glutamin(ROTH), 10\% FBS (heat-inactivated, Thermo Fisher) and 5ng/ml Growth Factor FGF-2 (PeproTech).

\section{Retroviral Transduction}
From 100\% confluent cells, seed cells at a ratio of 1:6 in T25 according to standard protocol. On the next day, remove medium, add 1ml of Polybren-Medium (MEM\textalpha{}, 5\% FBS, 1\mul{}/ml Polybrene). Carefully, add 1ml of virus construct to corresponding flask. Incubate at 4 \degC for two hours, followed by addition of 3ml Compensation-Medium (3ml MEM\textalpha{}, 5\% FBS, 1\mul{}/ml Polybrene and 8 ng/ml FGF-2). After letting it incubate at standard conditions overnight, remove medium and add 5ml standard growth medium. On the next day, split 1:3.5 depending on confluence. On the next day, start selection by removing old medium and adding Selection-Medium (Growth Medium + 3\textmu{}g/ml Puromycin) and incubating the cells for 72 hours. After incubation period has ended, pre-validate knock-out with comparing negative control (expected: low viability) to KO-cell-line (expected: high viability) under light microscope.\\
Definite validation of knock-out is done by both WesternBlot on Protein and qRT-PCR on mRNA. 

\section{RNA Analysis}
\subsection{RNA Extraction}
This was prepared in open space. Cells were either directly lysed in-well after PBS washing or lysed in the tube, after being collected from well and washed with PBS. Lysis-Buffer was supplemented with 1\% \textbeta-mercaptoethanol. The whole cell lysate was either frozen for later processing or directly processed. When freezing the lysate, the tube is shock-frozen in liquid nitrogen before transferred to -80 \degC freezer. When processing, the PureLink PCR Micro Kit (\product{Thermo Fisher, CN: K310250}) following manufacturer's protocol was used, dissolving the RNA in the final step with 20\mul of RNase-free Water(). The final RNA concentration was measured using the Qubit RNA HS Assay Kit (ThermoFisher) with a new calibration done every time a new working solution was mixed.

\subsection{Reverse Transcription}
This was prepared in open space. Unless stated otherwise, 300ng of RNA was transcribed per 40\mul total reaction volume using the High-Capacity cDNA Reverse Transcription Kit (\product{Thermo Fischer, CN: 4368813}) following supplier's instructions. Prepare general \textsc{MasterMix}(2X) consisting of dNTP, Random Primer, RT Buffer, MultiScribe and RNase-free Water. Mix 20\mul of \textsc{Master Mix} with 20\mul{} RNA in RNase-free Water. \myworries{Insert program specification of Cindy Thermal Cycler (MasterCycler, Eppendorf)}. Samples were either stored at -20 \degC or further processed immediately. 

\subsection{Quantitative Real-Time PCR Analysis}

All real-time PCR analyses were performed on an StepOnePlus \myworries{CHECK THAT!} ThermoCycler real-time PCR system (Applied Biosystems) with a 96-well plate as a carrier device. Each reaction well contained a reaction volume of 2\mul{} sample cDNA and 8\mul{} Primer-specific PCR MasterMix (KAPA PROBE FAST MasterMix (KAPA Biosystems), TaqMan\textregistered{} Primer and RNase-free Water). The amplification protocol consisted of \myworries{CHECK THAT}. The C\textsubscript{t}-value (cycle threshold) value for each gene was determined using the automated threshold analysis in the StepOnePlus device. Data was evaluated using the comparative C\textsubscript{t}-Method with GAPDH and RPL13A as housekeeping genes.

Investigated gene products and corresponding primers used are: 
\begin{itemize}
\item COL1
\item COL3
\item FN1
\item IL6
\item IL11
\item ALPL
\item RUNX1
\item SPP1
\item GAPDH 
\item RPL13A
\end{itemize}

\section{Western Blots}
\subsection{Preparation}
This was prepared in open space. 
Cell pellet was dissolved in 20\mul \textsc{Ripa}-Buffer and kept on ice during the whole time until cooking. After incubation on ice for 20 minutes, during which each sample was three to four times vortexed for $>$5 seconds, the samples are 4 \degC centrifuged at maximum speed for 10 minutes. Transfer supernatant, discard pellet. Measure protein concentration of all samples by employing the Bio Rad DC Protein Assay, following the manufacturer's protocol and comparing it to the personal standard curve \myworries{Link to Appendix}. Then relative protein content is normalized while maximized (i.e. dilute with RIPA all samples to match concentration of lowest concentration sample). 

\subsection{Western Blotting}
In new tube mix 13\mul{ }of Sample with 2.5\mul{ }of Laemmli-Buffer(6X) (Dilution-Error: $<$ 4\%), vortex 5 seconds and then cook for 4-8 minutes at 95 \degC in ThermoCycler (i.e. cooking). Then separate sample by SDS-PAGE (Used gel: 4–15\% Mini-PROTEAN TGX Stain-Free Protein Gel \product{Bio Rad, CN: 4568083}, constant current: 35 mA), then transferred onto PVDF Membrane. The membrane was then blocked in 5 wt\% low fat dry-milk in TBST (i.e. blocking solution) for 1 hour. This is followed by incubation primary antibody for 2 hours and subsequent incubation in matching secondary antibody for 1 hour. Every time after immersion medium changes, the membrane was washed three times in TBST for 10 minutes. Ultra enhanced chemiluscence HRP substrate \myworries{Find this CN} has been used to produce a signal in machine \myworries{What machine}. Exposure time was manually adjusted to produce a representative picture of the samples at hand. Quantification was done with Fiji (version 2.0.0-rc-69, Java 1.8) using the macro \myworries{What? }

\myworries{Here come up with a sexy representation. Maybe colour code Proteins of Interest}


\section{Chemical stimulation of \Piezo with \Yoda}
Experiment is initialised by employing a standard Trypsin-EDTA passaging protocol until after the centrifugation step. After centrifugation ended, the supernatant was discarded, the tube filled with 10ml sterile PBS, followed by another centrifugation step of 5 minutes at 500 rcf. Cell concentration was assessed right after transferring from Flask to Falcon Tube, using a Neubauer Improved Disposable Counting Chamber (\myworries{CHECKTHAT}). 
MSC's were seeded in \myworries{CHECK PLATE} at a density of 240'000 Cells/4cm\textsuperscript{2} in serum-free MEM$\alpha$, without any addition of FGF-2. 4 wells per time-point, corresponding for each possible combination of  intervention and read-out method (Yoda/Control and Protein/RNA, respectively). This is followed by a resting period of 8-14h in the incubator to allow for cell attachment, before progressing with the intervention.
Medium  and washed the cells with $> 2.5$ ml PBS and added the intervention medium (i.e. MEM$\alpha$ with either serum-free medium (negative control group) or \myworries{5\textmu{}M} \Yoda. Leave in incubator for 30 minutes. Afterwards wash twice, before adding MEM$\alpha$ again. Harvesting of both RNA and Protein samples after 0, 24, 48 and 72hours.\\ 
RNA samples were harvested by washing them with PBS, before adding lysis buffer with 1\% \textbeta-mercaptoethanol. Shock-freeze the lysate in liquid nitrogen. Harvesting of protein samples were done by washing with PBS, adding 0.5 ml of TE(1X) per well. Incubate it for 3-8 minutes. Add 1ml of MEM\textalpha{ }+10\% FBS per well, scrape the well surface with cell scraper (\myworries{Sarstedt, CN}) and transfer the mixture in sample tube. Centrifuge at 2'000 rcf for 5 minutes. Discard supernatant, add PBS, centrifuge again for 5 minutes at 2'000 rcf, then discard supernatant completely and shock-freeze tube in liquid nitrogen. Store samples in -80 deg C freezer.


\section{Fluidic Shear Stress Model}

This model allows for cell-agnostic seeding of adherent cells on grating with arbitrary feature space. It allows us to simulate fluidic shear stress of various magnitudes and with medium, that can be functionalized. During this project, we only quantified Ca2+-ion influx response, while cell recovery and subsequent biochemical analysis would also be possible. 

\subsection{Flowchamber and seeding}

First we combine silicone \myworries{Product Name} and cross-linker \myworries{Product Name} and mix it vigorously. Then put in vacuum chamber to rid the gel of all bubbles. Put the gel in negative, metal \myworries{form (?)}, then vacuum again. Then on heating plate over night at 70 deg Celsius. After the microscopy slides underwent plasma treatment, we glue a stamp on the slide using the silicon-crosslinker-mixture from earlier. Then we put them over night next to the metal form on a heating plate. On the next day, remove the PDMS-stamps, leaving behind PDMS grating. Decorate the stamps with Collagen-1 \myworries{Product Number} using Sulfo-SANPAH as a crosslinker. Cut the flowchamber pieces and punch entry point for syringe. Prepare glue by adding 3\mul of \myworries{Platin Reagent} to 1g of silicon, mix for 3 minutes. Add 0.1g of Crosslinker, mix again for 5 minutes. After 3 hours of incubation time, Collagen-1 cross-linking is finished. Now glue PDMS-pieces on glass slide and put in oven at \myworries{40 deg C} for 45 minutes. Either we stored them in the fridge or seed cells for an experiment on the next day.\\
For seeding, we work sterile and under aseptic conditions \myworries{?}. Flowchambers sterilized through generous use of 80\% EtOH (including flushing of chamber). Flush the chamber twice with PBS. Prepare cell solution of 1 Mio. Cells/ml (MEM\textalpha{ }with 10\% FBS and 5ng/ml FGF-2). Add approximately 70 \mul of cell solution per flowchamber. After 45 minutes flush the chamber to remove non-adherent cells. Put in incubator over night with the experiment scheduled on the next day.

\subsection{Fluorescence staining and imaging}
In 2ml of Growth Medium dissolve 5\mul{} Fluo-4 (\myworries{Product Details}) (i.e. staining medium). Define set of flowchambers (typically two), who are going to be imaged in 2 hours. Add 200\mul{} of staining medium per flow chamber and let it incubate until the start of the experiment. Minimise photobleaching by decreasing unnecessary light exposition. In a dark microscopy room, connect the flowchamber with a syringe that is fixed on the NemeSys syringe system (\myworries{Product Details}). The protocol used is stored in a \myworries{}. Imaging was done with chosen wavelength of \textlambda = 488 nm. \myworries{Excitation or Emission?} Imaging protocol included a Z-stack of 5 layers to account for shear-mediated displacement of cell. Reduce image to one dimensional projection with average intensity method.  \\
Note that in the extracellular Ca-influx experiment we flushed the flowchamber for 12 minutes (equivalent to 2.4ml medium) with normal ACSF in between measurements.

\myworries{ToDo: Name Software LiveAc, version number, ACSF-recipe, CaFree-ACSF-recipe}



\subsection{Processing and evaluation}

\begin{itemize}
\item n flowchambers per sample lead to pooled 3$\times$240 array, carrying [Mean, S(N), N]. No normalization between technical replicates with differing cell counts. 
\item One segmentation and normalisation per sample.
\item An archived version of the macros used can be accessed on a link.
\item Maximum Values for peak-comparison were found by doing a maximum-search on the whole measurement interval. (Leading to maximum before stimulus in CaFree-measurements in some samples.)
\end{itemize}


\section{Immuno-Staining and Confocal Microscopy}

