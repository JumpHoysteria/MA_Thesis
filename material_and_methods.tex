% Some commands used in this file
\renewcommand{\package}{\emph}
\newcommand{\product}{\textit}

\chapter{Material and Methods}

\section{General information}
All the liquids that will be in contact with the cell, are pre-warmed to limit cold-exposure.
\myworries{Write the sentence from Semester Project where I comment about where the products come from.}
More in-depth methods description can be found in the annex to this paper \myworries{TODO!}
Unless specifically stated otherwise, we worked under sterile conditions (Biosafety Level 1) inside the hood. 

\section{Generation of Knock-Out Cell Lines}
\myworries{Basically include KO Protocol, write only things necessary, without missing the most important part.}

\section{Cell Culture}
By courtesy of Dr. Ulrich Blache, we had access to human bone-marrow derived mesenchymal stem cells. They are grown in T75 Cell Culture Flasks (\product{Thermo Fisher, CN: 156499}) in an incubator with constant conditions of 37$^{\circ}$ C and 5 Percent CO2. Growth Medium consist of MEM$\alpha$ with nucleosides (\product{Thermo Fisher, CN: 22571020}), 10\% FBS and 5ng/ml Growth Factor FGF-2 (\myworries{Product Number}).

\section{qRT-PCR}

\section{Western Blots}

\section{Chemical stimulation of \Piezo with \Yoda}

First we seeded hBM-MSC (120'000 Cells/ml, 2 ml per 9cm\textsuperscript{2}-Well ), with only MEM$\alpha$, serum free and without FGF-2 and left them in the incubator \myworries{TODO} to let them adhere to the surface for 8-14h. Half an hour before intervention, we incubated an adequate amount of medium in the incubator, to limit differences of dissolved gases. Afterwards, we discarded the medium, washed the cells with $> 2.5$ ml PBS and added the intervention medium (i.e. MEM$\alpha$ with either nothing (Control Group) or \myworries{5uM} Yoda1 (Piezo1-Activation Group)). Transfer to incubator for 30 minutes. Afterwards wash twice with PBS and then add MEM$\alpha$ again. Harvesting after n$\times$24h, with n = 0,1,2 and 3. Exception: n = 0 (Day 0 Group) has only control group. 


\section{Fluidic Shear Stress Model}

\subsection{Flowchamber and seeding}

\subsection{Fluorescence Staining}



\subsection{Processing and evaluation}

\section{Features}

