\chapter{Limitations \& Outlook}

\section{Limitations}
One of the biggest limitation is that we relied on chemical stimulation of \Piezo{} for the majority of this study by making use of the selective \Piezo{} agonist \Yoda{}. To our best knowledge, there is no accepted transformation that relates concentration and exposure duration of \Yoda{} to an equivalent mechanical force. This challenging endeavour could corroborate the validity of \Yoda{} studies and potentially leapfrog \Piezo{} research. While we adhered to standard practices regarding chemical activation of \Piezo{}\cite{Morley2018}, it is still critical to reproduce the results with mechanical stimulation alone to validate the (patho-)physiological relevance of the effects observed. DEVELOP The in vitro nature of chosen study design may affect PIEZO1 expression. 
Another limitation of this study is the small sample size throughout all experiments. An increase thereof could improve statistical significance and sensitivity for more subtle effects.\\




In addition to Piezo1, alternative shear stress sensing mechanisms have been suggested.


\begin{itemize}
    \item Long Term: They start to grow again, but collagen1 remains low

\end{itemize}

McHugh and colleagues report reduced calpain activity in knock-down of Piezo1 [McHugh 2011], calpain being a Ca\textsuperscript{2+} dependent protease. [Goll2003]. If it would be shown that increased Piezo1 activity could also lead to increased Calpain activity (the mirrored case of McHugh's report), then this could be a second explanation.
