\chapter{Limitations \& Outlook}

\section{Limitations}
One of the biggest limitation is that we relied on chemical stimulation of \Piezo{} for the majority of this study by making use of the selective \Piezo{} agonist \Yoda{}. To our best knowledge, there is no accepted transformation that relates concentration and exposure duration of \Yoda{} to an equivalent mechanical force. This challenging endeavour could corroborate the validity of \Yoda{} studies and potentially leapfrog \Piezo{} research. While we adhered to standard practices regarding chemical activation of \Piezo{}, it is still critical to reproduce the results with mechanical stimulation alone to validate the (patho-)physiological relevance of the effects observed. 
Another limitation of this study is the small sample size throughout all experiments. An increase thereof could improve statistical significance and sensitivity for more subtle effects.\\
The scope of this study was unidirectional. This means we investigated how \Piezo{} acts on MSCs, despite the evidence of a more bi-directional relationship between channel and cell as it has been shown that \Piezo{} activity can be modified depending on cell state, which by itself is a function of intra- and extracellular processes. A more comprehensive project could assess the main \textit{in vivo} influences on MSCs and then try to incorporate them, for example with co-culturing or a variable load study design.


\begin{itemize}
    \item Long Term: They start to grow again, but collagen1 remains low

\end{itemize}

McHugh and colleagues report reduced calpain activity in knock-down of Piezo1 [McHugh 2011], calpain being a Ca\textsuperscript{2+} dependent protease. [Goll2003]. If it would be shown that increased Piezo1 activity could also lead to increased Calpain activity (the mirrored case of McHugh's report), then this could be a second explanation.