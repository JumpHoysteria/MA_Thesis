\chapter{Outlook}
In this work we produce insight indicating contribution of \Piezo{} in MSC mechanosensing and provide evidence for \Piezo{} being implicated in ECM turnover, which could have potential implications on the stem cell niche.


In the future, the results generated through chemical stimulation with \Yoda{}, have to be replicated with purely mechanical stimuli. To the best of our knowledge, there is no accepted transformation that relates concentration and exposure duration of \Yoda{} to an equivalent mechanical force. We relied on chemical stimulation of \Piezo{} for the majority of this study by making use of the selective \Piezo{} agonist \Yoda{}. While we adhered to standard practices regarding chemical activation of \Piezo{}\cite{Morley2018}, it is still critical to reproduce the results with mechanical stimulation alone to validate the physiological relevance of the \textit{in vitro} effects observed. \par

For future experiments it might help to increase the sample size, to improve sensitivity for more subtle effects and statistical power overall. \par

The findings of this work suggest there are still a lot of potential topics worth investigating in the relationship between \Piezo{} and MSCs. We propose two main directions in which further research could go:\\
Firstly, the molecular pathways downstream of \Piezo{} could be investigated more thoroughly. The cytosolic calcium-dependent protease calpain might be a promising target. As discussed in chapter ~\vref{sec:PiezoandECM} calpain activity as a result of \Piezo{} was already described in epithelial cells \cite{McHugh2010}. Studies where we alter calpain-activity (e.g. with calpain-inibitor calpeptin \cite{Schoenwaelder1999})) or deprive the cells of extracellular calcium, could potentially give insight into the molecular pathways underpinning the effects observed in MSCs. Another potential target of \Piezo{} downstream signalling is the nuclear factor YAP\textbackslash{}TAZ. YAP\textbackslash{}TAZ are nuclear factors that exhibit their function through intracellular location (i.e. nuclear or cytosolic) and are known to be heavily implicated in mechanotransduction \cite{Dupont2011}. Pathak and colleagues already succeeded in producing an account of YAP to be downstream of \Piezo{}, as siRNA \Piezo{}-knockdown in neural stem/progenitor cells overrode the YAP-mediated influence of physical cues like stiffer substrate, effectively altering lineage choice \cite{Pathak2014}. Continuous tracing of YAP/TAZ complex (e.g. with time-lapse microscopy) in response to \Yoda{}-activation and/or altered \Piezo{}-expression, could produce seminal insight.\par

The second suggested direction of future research is the phenotypic description of \Piezo{} activation in MSCs. Regarding the experiment in which we investigated intracellular protein levels of \Piezo{}-activated cells three days after intervention, we could try to gain more insight about the onset and duration of the effect by increasing the temporal resolution and the experimental time-frame. This data could then maybe lead us to an explanation of the observations at hand and potentially help us in coming closer to successful application of MSC in personalized medicine.

Heavy posttranscriptional modification of Collagen Triple Helix in MSC mediated through Piezo1? If this should be true and load actually decreases protein, then this finding could have direct implications on therapy recommendation and possibly future research. https://www.sciencedirect.com/science/article/pii/B9780123864710000092?via%3Dihub


DRAFT STAGE: 
In this work we showed \Piezo{} is important in MSC mechanosensing. It has already been shown that mechanical loading leads sensitivity through TRPV-4.  Another effect that we saw is that Piezo activation leads to degradation of ECM components and possible has implications on differentiation. Also architecture and composition of extracellular space.
As for future directions it would be important to identify the patways involved to paint a more complete picture of MSC mechanosensing. Sugimoto shows that Piezo1 leads to osteogenic differentiation. Wang2018 showed that Piezo1 activate YAP/TAZ nuclear localization. Time-lapse microscopy or culturing on soft substrate and try to override substrate influence through Piezo1-stimulation. Calpain could also be activated through Piezo1 dependent influx of calcium. MAP-Kinase could also be activated in turn.  Check whether effects are really calcium dependent, repeat Yoda-experiment with calcium starved MEMalpha or EGTA supplemented MEMalpha. If effect persists then effect is likely calcium-independent. If effect vanishes without calcium, study with calpain inhibitor. Otherwise, XMU-MP-1 (Hippo-Pathway inhibitor) that should inhibit YAP nuclear localization. To describe the effect better, we could also widen our analysis and try to describe the effect in Vivo. Either like Morley et al 2019 who injected it into mice and looked at tissue or analysing MSC-niche in GOF mutants. Alternatively we could also try to mimic near-physiological conditions with co-culturing.
McHugh and colleagues report reduced calpain activity in knock-down of Piezo1 [McHugh 2011], calpain being a Ca\textsuperscript{2+} dependent protease. [Goll2003]. If it would be shown that increased Piezo1 activity could also lead to increased Calpain activity (the mirrored case of McHugh's report), then this could be a second explanation.
*Setting insight in recent study context, that could show how MSC could help in tendinopathy but still fail* Further studies could help us to actually proceed and do really cool things with the power of MSCs. 