\chapter{Outlook}
In this work we produce insight indicating contribution of \Piezo{} in MSC mechanosensing and provide evidence for \Piezo{} being implicated in ECM turnover, which could have potential implications on the stem cell niche.


In the future, the results generated through chemical stimulation with \Yoda{}, have to be replicated with purely mechanical stimuli. To the best of our knowledge, there is no accepted transformation that relates concentration and exposure duration of \Yoda{} to an equivalent mechanical force. We relied on chemical stimulation of \Piezo{} for the majority of this study by making use of the selective \Piezo{} agonist \Yoda{}. While we adhered to standard practices regarding chemical activation of \Piezo{}\cite{Morley2018}, it is still critical to reproduce the results with mechanical stimulation alone to validate the physiological relevance of the \textit{in vitro} effects observed. \par

For future experiments it might help to increase the sample size, to improve sensitivity for more subtle effects and statistical power overall. \par

The findings of this work suggest there are still a lot of potential topics worth investigating in the relationship between \Piezo{} and MSCs. We propose two main directions in which further research could go:\\
Firstly, the molecular pathways downstream of \Piezo{} could be investigated more thoroughly. The cytosolic calcium-dependent protease calpain might be a promising target. As discussed in chapter ~\vref{sec:PiezoandECM} calpain activity as a result of \Piezo{} was already described in epithelial cells \cite{McHugh2010}. Studies where we alter calpain-activity (e.g. with calpain-inibitor calpeptin \cite{Schoenwaelder1999})) or deprive the cells of extracellular calcium, could potentially give insight into the molecular pathways underpinning the effects observed in MSCs. Another potential target of \Piezo{} downstream signalling is the nuclear factor YAP\textbackslash{}TAZ. YAP\textbackslash{}TAZ are nuclear factors that exhibit their function through intracellular location (i.e. nuclear or cytosolic) and are known to be heavily implicated in mechanotransduction \cite{Dupont2011}. Pathak and colleagues already succeeded in producing an account of YAP to be downstream of \Piezo{}, as siRNA \Piezo{}-knockdown in neural stem/progenitor cells overrode the YAP-mediated influence of physical cues like stiffer substrate, effectively altering lineage choice \cite{Pathak2014}. Continuous tracing of YAP/TAZ complex (e.g. with time-lapse microscopy) in response to \Yoda{}-activation and/or altered \Piezo{}-expression, could produce seminal insight.\par

The second suggested direction of future research is the phenotypic description of \Piezo{} activation in MSCs. Regarding the experiment in which we investigated intracellular protein levels of \Piezo{}-activated cells three days after intervention, we could try to gain more insight about the onset and duration of the effect by increasing the temporal resolution and the experimental time-frame. This data could then maybe lead us to an explanation of the observations at hand and potentially help us in coming closer to successful application of MSC in personalized medicine.