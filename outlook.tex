\chapter{Limitations \& Outlook}

\section{Limitations}
One of the biggest limitation is that we relied on chemical stimulation of \Piezo{} for the majority of this study by making use of the selective \Piezo{} agonist \Yoda{}. To our best knowledge, there is no accepted transformation that relates concentration and exposure duration of \Yoda{} to an equivalent mechanical force. This challenging endeavour could corroborate the validity of \Yoda{} studies and potentially leapfrog \Piezo{} research. While we adhered to standard practices regarding chemical activation of \Piezo{}\cite{Morley2018}, it is still critical to reproduce the results with mechanical stimulation alone to validate the (patho-)physiological relevance of the effects observed. DEVELOP The in vitro nature of chosen study design may affect PIEZO1 expression. 
Another limitation of this study is the small sample size throughout all experiments. An increase thereof could improve statistical significance and sensitivity for more subtle effects.\\




In addition to Piezo1, alternative shear stress sensing mechanisms have been suggested.


\begin{itemize}
    \item Long Term: They start to grow again, but collagen1 remains low

\end{itemize}


\section{Outlook}
DRAFT STAGE: 
In this work we showed \Piezo{} is important in MSC mechanosensing. It has already been shown that mechanical loading leads sensitivity through TRPV-4. Facultative: Patapoutian made a compelling point for Piezo1 expression confounding measured mechanosensitivity. Another effect that we saw is that Piezo activation leads to degradation of ECM components and possible has implications on differentiation. Also architecture and composition of extracellular space.
As for future directions it would be important to identify the patways involved to paint a more complete picture of MSC mechanosensing. Sugimoto shows that Piezo1 leads to osteogenic differentiation. Wang2018 showed that Piezo1 activate YAP/TAZ nuclear localization. Time-lapse microscopy or culturing on soft substrate and try to override substrate influence through Piezo1-stimulation. Calpain could also be activated through Piezo1 dependent influx of calcium. MAP-Kinase could also be activated in turn.  Check whether effects are really calcium dependent, repeat Yoda-experiment with calcium starved MEMalpha or EGTA supplemented MEMalpha. If effect persists then effect is likely calcium-independent. If effect vanishes without calcium, study with calpain inhibitor. Otherwise, XMU-MP-1 (Hippo-Pathway inhibitor) that should inhibit YAP nuclear localization. To describe the effect better, we could also widen our analysis and try to describe the effect in Vivo. Either like Morley et al 2019 who injected it into mice and looked at tissue or analysing MSC-niche in GOF mutants. Alternatively we could also try to mimic near-physiological conditions with co-culturing.
McHugh and colleagues report reduced calpain activity in knock-down of Piezo1 [McHugh 2011], calpain being a Ca\textsuperscript{2+} dependent protease. [Goll2003]. If it would be shown that increased Piezo1 activity could also lead to increased Calpain activity (the mirrored case of McHugh's report), then this could be a second explanation.
*Setting insight in recent study context, that could show how MSC could help in tendinopathy but still fail* Further studies could help us to actually proceed and do really cool things with the power of MSCs. 