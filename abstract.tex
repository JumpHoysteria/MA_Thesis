\begin{abstract}
The influence of mechanical forces on cellular dynamics is ubiquitous and thus fundamental. Examples include tendon and muscle cells that adapt to repeated physical stimuli or stem cells that commit to their differentiation depending on substrate properties. When cells and tissues are subjected to mechanical stresses, various pathways are activated, which then transduce the signal, coming together in a finely orchestrated response. This intricate interplay of multiple pathways in parallel necessitates studies on isolated influence of single contributors in order to overcome limitation in scientific understanding and medical application. Mesenchymal stem cells (MSCs) are a great example as they hold immense potential in regenerative medicine, but applications up until now have been shown to be either ineffective or insecure. Furthermore, MSC are known for being susceptible to physical stimuli, however the contribution of individual pathways is not fully understood. In this study, we investigate the influence of the ion-channel \Piezo{} on MSCs. The trans-membrane mechanically gated channel \Piezo{} opens its central pore upon activation, consequently allowing calcium to enter the cytosol and thus to propagate the signal. Through combination of a versatile fluidic shear stress model and widefield calcium imaging, we identify the significance of and gain mechanistic insight in \Piezo{} mechanosensing in MSC. Subsequently, we study effects of \Piezo{} activation on gene expression and protein content over time by making use of the \Piezo{}-specific agonist \Yoda{}. Finally, investigate \Piezo{} influence on osteogenic differentiation. 
In summary, in this study we investigate the neglected contribution of \Piezo{} in MSC mechanosensing, present implications on ECM homeostasis and finally discuss the resulting impact on the stem cell niche.
\end{abstract}
