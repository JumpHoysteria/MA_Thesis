\begin{abstract}
The influence of mechanical forces on cellular dynamics is ubiquitous and thus fundamental. Examples include tendon and muscle cells that adapt to repeated physical stimuli or stem cells that commit to their differentiation depending on physical substrate properties. When cells and tissues are subjected to mechanical stresses, downstream pathways are activated, which then transduce the signal resulting in cellular responses. This intricate interplay of multiple pathways simultaneously, necessitates studies on isolated influence of single contributors in order to gain insight for overcoming limitation in scientific understanding and medical application. Mesenchymal stem cells (MSCs) are a good example as they hold great potential in regenerative medicine, but applications up until now have been shown to be either ineffective or unsafe. Furthermore, MSC are known for being susceptible to physical stimuli, however the contribution of individual pathways is not fully understood. In this study, we investigate the influence of the channel \Piezo{} on MSCs. The trans-membrane mechanically gated channel \Piezo{} opens its central pore upon activation, thus allowing calcium ions to enter the cytoplasm and to propagate the signal. Through combination of a fluidic shear stress setup and live calcium imaging, we identify the significance of and gain mechanistic insight in \Piezo{} mechanosensation in MSC. Subsequently, we study effects of \Piezo{} activation on gene expression and protein content over time by making use of the \Piezo{}-specific pharmacological  agonist \Yoda{}. Finally, we investigate \Piezo{} influence on osteogenic differentiation. 
In summary, we investigate the contribution of \Piezo{} to MSC mechanotransduction, present implications on ECM homeostasis and finally discuss the resulting impact on the stem cell niche.
\end{abstract}
