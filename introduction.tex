% Some commands used in this file
\newcommand{\package}{\emph}


\chapter{Introduction}

\section{Introduction}
\subsection{Motivation and Aim}
\label{sec:motivation}
Mechanosensation describes the phenomenon of cells sensing external physical stimuli. \myworries{Add an example, like aligning cells in groove\dots} This modality adds to the pre-existing paradigm of biological and chemical influences that cells can sense and react to. Mechanosensation describes one aspect of mechanobiology, which describes next to the physical stimuli, cells react to, the reaction of a cell to external physical stimuli on every level (i.e. from sensing to changing gene expression profiles).  With new publications almost weekly and new journals emerging, it becomes clear that mechanobiology is of far reaching consequences and very complex, with a multitude of biological sciences affected from ground up. 
\par 

Unquestionably, mechanobiology promises great advances in the field of regenerative medicine. This relatively new field of biomedical research is driven by the vision to restore the physiological status in a patient, by means of supporting endogenic restorative capabilities and/or implantation of biogenic substitutes often with cellular components that help to ameliorate the condition. \myworries{Insert Example} Even though the intentions are clear, there are still many open questions, that hinder systematic success. Therefore, it is imperative to gain a thorough understanding of the target system and its development\myworries{emergence, embryological origin}. For this reason, research in regenerative medicine is heavily implicated with stem cells. Adding to this reason, stem cells, by their nature, have reproductive and differentiating potential. This versatile quality render them pivotal to therapies emerging from regenerative medicine. To develop such a therapy, we need a better understanding of the system. \par

\subsection{Hypothesis and Goal}




\section{Theory}

\subsection{Piezo1}
Described in 2010 for the first time, Piezo1 is a  mechanically activated ion channel abundantly expressed in all eukaryotes.\cite{Coste55} \\
Three different states (active, inactive and non-reactive)\\
Blade Structure, Multipass Protein, Effector unit and sensor unit\\
Peculiar structure and recent discovery, make it a very interesting object of study. 

\subsection{Yoda1}
Piezo1-selective Agonist.
Botello and colleagues suggest a model wherein Yoda1 acts as molecular wedge, facilitating force-induced conformational changes, effectively lowering the channel's mechanical threshold for activation.'\cite{BotelloSmith.2019} Also is unstable in physiological conditions as we will discover in ~\vref{sec:biostability} 

\subsection{Bone Marrow-derived Mesenchymal Stem Cells}
The CDXX++/CDXX--/CDXX++ mesenchymal stem cells are produced during embryogenesis, they go through\dots Until they finally become the pre-cells to\dots 
While there are different types of cells that could be understood to be mesenchymal stem cells, with differences regarding potency and hematopoietic potential, in this work we will focus entirely on human bone-marrow derived mesenchymal stem cells. 

