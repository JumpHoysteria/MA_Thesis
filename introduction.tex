% Some commands used in this file
\newcommand{\package}{\emph}


\chapter{Introduction}

\section{Introduction}
\subsection{Motivation and Aim}
\label{sec:motivation}
Regenerative medicine is a relatively new field of biomedical research, which is driven by the vision to restore the physiological status in a patient, by means of supporting endogenic restorative capabilities and/or implantation of biogenic substitutes often with cellular components that help to ameliorate the condition. \myworries{Insert Example}. Regenerative medicine is relevant and new discoveries needed. Market of XXX Mio. \$. A lot of patients and potential. Examples are one, two and three. However, there is one that deserves special mention: Tendon reconstruction. Performed every year X amount of times, with current patient satisfaction at X \%. Even though the goal is clear, there are still many open questions, that hinder systematic success. Since XXXX, there are studies being done were cells are co-transplanted to ameliorate healing and thought to eventually lead to a better patient outcome. This strategy is derived from the school of thought, that cells can direct other cells behaviour. Either by direct interaction (e.g. by secretion of cytokines or para-hormones) or through secondary pathways like orchestrating ECM-composition that in turn helps in the coordination of cellular behaviour and implant biointegration. Stem cells are often the preferred choice, since they have the property of self-renewal and a lot of differentiation potential. 
\myworries{Short comparison of different cell types. MSC chosen most often because..} \myworries{Immunomodulatory and relative abundance?} This versatile qualities render them pivotal to therapies emerging from regenerative medicine. Still a lot of open questions of the inner workings from MSC's. One of the questions is what channels are involved in physical force sensing in MSCs and how they react to it. 
Piezo1 has been described as almost universally expressed mechanosensitive ion channel.  Piezo1 is a mechanosensitive ion channel and its discovery in 2009 marks a groundbreaking advancement in cell biology as it ends the search for the first mechanosensitive ion channel found in eukaryotic cells. 
The existence of mechanically gated (i.e. mechanosensitive) ion channels was already postulated in 1950 in a study measuring a change of electric potential in response to stretching of muscle. By now we know that mechanosensitive ion channels (MechanIC’s) underlie a lot of physiological functions, like regulating the myogenic tone of resistance arteries in muscle or stimulation of vasodilatation in the vascular system. Especially, in pain research, were neurons implicated in pain sensation seem to rely on MechanIC’s, the necessity for discovery of a mechanosensitive ion channel is given. Notwithstanding the relevance, the discovery of the first gene encoding a MechanIC would take another 44 years, until Sukharev and colleagues describe two distinct genes mscL and mscS, whose product mediate stretch sensing in the membrane of E. Coli bacteria. While this and following studies gave great insight into structure and biophysical mechanism of those channels, eukaryotic homologs were still missing. Until recently, the only eukaryotic candidate was TREK-1, a potassium-selective ion channel, whose mechanosensitive property is still controversial. In their breakthrough paper from 2009, Coste and colleagues report the discovery of two genes Fam38A and Fam38B, which encode mechanosensitive protein Piezo1 and Piezo2, respectively. Even though they carefully avoided calling Piezo1 an ion channel in their first paper, a follow-up study in the same lab confirmed piezo proteins as subunits for the homo-trimeric MechanIC Piezo1. Finally, scientists characterized the first eukaryotic mechanosensitive ion channel. 

In this work, by making use of KO-cell lines and Piezo1-specific agonist Yoda1, we investigate the effect of Piezo1 on ECM-homeostasis in mesenchymal stems cells. 




\section{Theory}

\subsection{Piezo1}
Described in 2010 for the first time, Piezo1 is a  mechanically activated ion channel abundantly expressed in all eukaryotes.\cite{Coste55} \\
different states (active, inactive [and non-reactive])\\
Blade Structure, Multipass Protein, Effector unit and sensor unit\\
Peculiar structure and recent discovery, make it a very interesting object of study. 

\subsection{Yoda1}
Piezo1-selective Agonist.
Botello and colleagues suggest a model wherein Yoda1 acts as molecular wedge, facilitating force-induced conformational changes, effectively lowering the channel's mechanical threshold for activation.'\cite{BotelloSmith.2019} Also it seems to have limited biostability in physiological conditions as we will discover in ~\vref{sec:biostability} 

\subsection{Bone Marrow-derived Mesenchymal Stem Cells}
While there are different types of cells that could be understood to be mesenchymal stem cells, with differences regarding potency and hematopoietic potential, in this work we will focus entirely on human bone-marrow derived mesenchymal stem cells. Those cells are defined by their property of being able to 

