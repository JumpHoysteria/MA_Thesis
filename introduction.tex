\chapter{Introduction}

\section{Introduction}
\subsection{Motivation}
\label{sec:motivation}
Mesenchymal stem cells will disrupt the medicine of tomorrow and will become a routine biomedical intervention in the rising world of personalized medicine. 
Mesenchymal stem cells (MSC) can give rise to various daughter cells found in fat- (adipocytes), cartilage- (chondrocytes), bone- (osteoblasts) and tendon-tissue (tenocytes) \cite{Barlow2008, Hass2011}. Because of this versatility, their pronounced capacity for self-renewal and their robustness, they are subject to different studies regarding potential applications in medical therapy like cardiac repair \cite{Pittenger2004}, spinal regeneration \cite{Goldschlager2010} and tendon replacement therapy.  
Tendon is collagenous and highly tensile tissue that connects muscle to bone, acting as a force transmission and effectively enabling us to move. Add Comment about tendon metabolism. The well-being of our tendon is crucial, as patients with tendon-related diseases suffer a lot and they are increasing in number with a projected amount of XX Patients in 20XX. The most prevalent tendon related disease next to rupture is called tendinopathy and describes an inflammatory disease of the tendon, with most common symptoms reported being movement associated pain, swelling and substantial decrease in quality of life. On a clinical level erroneous blood vessel formation inside the normally non-vascularized tendon core (vascular in-growth) is the hallmark of this disease. Currently, the pathophysiology is poorly understood. This lack of understanding of tendinopathy etiology leads to the treatment being largely symptomatic.  Some cases progress and make surgical intervention necessary. In those cases, the surgeon can cut out the diseased part, hence the need for tendon replacement surgery.  One approach is to transfer tendon from a non-diseased site. *Explain Problems that we are left with and how mesenchymal stem cells can help us* They hold great promises, but systematic success is still lacking. This is largely due to undefined behaviour. *Explain reasons for why they do not work now* More and more, it becomes clear that we need to understand the dynamics better that underlie MSC’s before we can start utilizing their full potential. While basic research on MSC has greatly increased our knowledge, we now realize that Biology and Chemistry alone are not enough to explain the cell. We also need to account for Physics as it becomes clearer that mechanosensing, the ability to “feel forces”, is an underlying factor in almost all cellular contexts, from migration over function to differentiation and self-renewal.  
In a landmark study LALALA et al. showed that stem cells can decide on their cell fate only due to physical cues, like feature size of their environment and substrate stiffness. Next to a remarkable discovery by itself, this study is yet another account showing that there is new, decisive knowledge in studying biophysical interaction between cells and their physical surroundings since the fundamental pathways that govern this dynamic are only described incompletely if at all. Mechanobiology is the domain of biology at the interface between cells and their physical environment. Even though the first hypothesis that involve cells being able to sense physical stimuli dates back to the late 19th century, mechanobiology is only now gaining a lot of scientific momentum with an average of nine new papers accessible on PubMed per week over the last 5 years. Eyckman and colleagues explain the delay from idea to proof of existence mainly due to lacking technology. Others argue that researches simply underestimated the reach and potential consequences. Contemporarily, much research is generating groundbreaking insight. We now know that mechanical cues drive tissue homeostasis, hearing sensation, tumour formation.  One of the most influential ones, was the identification of mechanically gated channel (MechanIC) named Piezo1, which ends the 80-year-old search for the unit on the interface.
Piezo1 is a mechanosensitive ion channel and its discovery in 2009 marks a groundbreaking advancement for mechanobiology and cell biology in general as it is the first description of a mechanosensitive ion channel found in eukaryotic cells. The existence of mechanically gated ion channels was already postulated in 1950 in a study measuring a change of electric potential in response to stretching of muscle. By now we know that mechanosensitive ion channels (MechanIC’s) underlie a lot of physiological functions, like regulating the myogenic tone of resistance arteries in muscle or stimulation of vasodilatation in the vascular system. Especially, in pain research, were neurons implicated in pain sensation seem to rely on MechanIC’s, the necessity for discovery of a mechanosensitive ion channel is given. Notwithstanding the relevance, the discovery of the first gene encoding a MechanIC would take another 44 years, until Sukharev and colleagues describe two distinct genes mscL and mscS, whose product mediate stretch sensing in the membrane of E. Coli bacteria. While the efforts and papers, that were inspired by this discovery, gave great insight into structure and biophysical mechanism of those channels, eukaryotic homologs were still missing. Until recently, the only eukaryotic candidate was TREK-1, a potassium-selective ion channel, whose mechanosensitive property is still controversial. Finally, in their breakthrough paper from 2009, Coste and colleagues report the discovery of two genes Fam38A and Fam38B, which encode mechanosensitive protein Piezo1 and Piezo2, respectively. Even though they carefully avoided calling Piezo1 an ion channel in their first paper, a follow-up study in the same lab confirmed piezo proteins as subunits for the homo-trimeric MechanIC Piezo1. Finally, scientists characterized the first eukaryotic mechanosensitive ion channel. Conveniently, the same lab would also report the discovery of a Piezo1-specific agonist, named Yoda1 (a nod at Filmmaker George Lucas’ use of “the Force”), which greatly facilitates Piezo1 resarch. \cite{Syeda2015} Due to massive biophysical studies regarding both Piezo1 \cite{Saotome2018} and its Yoda1-interaction \cite{Lacroix2018} we now have a better picture of how the mechanism by which Yoda1 activates Piezo1 \cite{Botello-Smith2019}. This enables a plethora of potential ideas, in which we can probe the influence of Piezo1 in different settings.
By serendipity, protein mass spectroscopy of conditioned media from cells whose Piezo1 has been activated, showed decreased In this work, by making use of KO-cell lines and Piezo1-specific agonist Yoda1, we investigate the effect of Piezo1 on ECM-homeostasis in mesenchymal stems cells.
This technical advances has increased our understanding. Also stimulated fresh examination of mechanics with important consequences.


To be deleted 
\begin{itemize}
    \item Regenerative medicine is a relatively new field of biomedical research, which is driven by the vision to restore the physiological status in a patient, by means of supporting endogenic restorative capabilities and/or implantation of biogenic substitutes often with cellular components that help to ameliorate the condition.The average person is always getting older and age-associated disease will become more prevalent. This increases pressure for medical therapies that address them. The world wide market for regenerative market is said to grow 20\% each year, and is projected to reach around USD 73 billion by the year 2027.[MarketsAndMarkets, https://cutt.ly/WrUSspf] The growing tendency can be explained because of the paradigm of regenerative therapy, which is restoration and reconstitution of patient. Some examples are cardiac repair \cite{Pittenger2004} and applications in spine-surgery \cite{Goldschlager2010}. However, there is one that deserves special mention: Mesenchymal Stem cell applications to tendon healing. Even though the goal is clear, there are still many open questions, that hinder systematic success. Since XXXX, there are studies being done were cells are co-transplanted to ameliorate healing and thought to eventually lead to a better patient outcome. This strategy is derived from the school of thought, that cells can direct other cells behaviour. Either by direct interaction (e.g. by secretion of cytokines or para-hormones) or through secondary pathways like orchestrating ECM-composition that in turn helps in the coordination of cellular behaviour and implant biointegration. Stem cells are often the preferred choice, since they have the property of self-renewal and a lot of differentiation potential. 
\myworries{Immunomodulatory and relative abundance?} This versatile qualities render them pivotal to therapies emerging from regenerative medicine. Still a lot of open questions of the inner workings from MSC's. One of the questions is what channels are involved in physical force sensing in MSCs and how they react to it. 
\end{itemize}

\subsection{Aim}


\section{Hypothesis}

\begin{itemize}
	\item Sass
	\item 
\end{itemize}


\section{Theory}

\subsection{\Piezo{}}

\myworries{ADD NICE PIEZO1-Picture!}

\Piezo{} and \textsc{Piezo2} are the only members of the \textsc{Piezo}-family. The protein family is genetically and structurally unique, as the similarity to other proteins are minimal \cite{Coste2010}. In this work we only focus on \Piezo{}, which constitutes the mechanosensitive homo-trimeric ion channel \Piezo{}, which has a unique architecture resembling a propeller with three blades, where each of those blades are constituted by one \Piezo{} Protein. Functionally, the channel can be described the central pore, the anchor and beam region and the outer leaflets or beams. 

\begin{itemize}
	\item different views on count of transmembrane regions Saotome2018
	\item Anchor and Beam Connect Blade to conducting pore Saotome2018
	\item Hyphothesis that actually force transduction from blade to pore is opening mechanism, but the exact biophysical mechanism remains elusive. 
\end{itemize}

 When in an open state, it selectively permits passage of extracellular calcium, which can then acts as a cofactor further downstream. \cite{Coste2010}. The channel is expressed almost ubiquitously in our body. Furthermore, \Piezo{} is evolutionary preserved with analogues spanning multiple evolutionary kingdoms including plants. \cite{Coste2010}\\
Although, in literature the name \Piezo{} is used interchangeably for the channel itself as also the protein that constitutes the channel, in this work we will from now one refer to the channel as \Piezo{}.\\
\Yoda{} is a small molecule, able to chemically activate \Piezo{} selectively \cite{Syeda2015}. The proposed mechanism is that Yoda1 acts as a molecular wedge, lowering the mechanical threshold for activation \cite{Botello-Smith2019}. This allows Yoda1 to partially activate \Piezo{} in absence of externally applied forces \cite{Lacroix2018}.

\subsection{Bone Marrow-derived Mesenchymal Stem Cells}
MSC's are characterized by four different properties: First, they hold the potential to differentiate into a defined set of daughter cells, including osteocyte (bone), chondrocyte (cartilage), adipocyte (fat-tissue) and tenocyte (tendon).\cite{Ng2008} Second, they hold the potential to self-renew, meaning they can give rise to identical cells. Third, they expose a set of stem cell markers like CD31, CD34 and cKit. While those markers are associated with stem cells, it is important to keep in mind that they are not unique to them. \cite{Battula2009} And fourth, they can be conveniently isolated from primary tissue based on their capacity to stick on plastic. \cite{Buhring2007}. Furthermore, there is a growing body of evidence that attributes an immunomodulatory effect of MSC on injured or inflamed sites, which could relate to better outcome in medical intervention \cite{Hass2011, Caplan2011}.
While there are differing anatomical sources where mesenchymal stem cells can be isolated from, including the umbilical cord and adipose tissue \cite{Barlow2008, Hass2011} in this work we are going to focus on human bone-marrow derived mesenchymal stem cells and from now on will refer to them as mesenchymal stem cells (MSCs).

\subsection{Shear force mechanosensing}
Definition of shear forces. Locate them in different physical forces in the body (stretch, compression,) Distinguish also from passive physical properties, that are then explored by the cell (e.g. substrate stiffness, topology).
Physiological context (different shear: fluidic vs condensed)
Although, in literature the name \Piezo{} is used interchangeably for the channel itself as also the protein that constitutes the channel, in this work we will from now one refer to the channel as Piezo1.

