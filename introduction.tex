% Some commands used in this file
\newcommand{\package}{\emph}


\chapter{Introduction}

\section{Motivation}
\label{sec:motivation}

In this Master's Thesis, we investigate the influence of Piezo1-stimulation on ECM Production in stromal cells.

\section{Piezo1}
Described in 200X for the first time, Piezo1 is a  mechanically activated ion channel abundantly expressed in all eukaryotes including plants.\cite{Coste55} \\
Three different states (active, inactive and non-receptive)\\
Blade Structure, Multipass Protein, Effector unit and sensor unit, highly conserved across species\\
Peculiar structure and evolutionary properties, next to abundance and recent finding, make it a very interesting object of study. 

\section{Yoda1}


\section{Mesenchymal Stem Cells}

\section{Hypothesis}

\section{Aim}

\section{Features}
\label{sec:features}

The template is divided into \TeX{} files as follows:
\begin{enumerate}
\item \texttt{thesis.tex} is the main file.
\item \texttt{extrapackages.tex} holds extra package includes.
\item \texttt{layoutsetup.tex} defines the style used in this document.
\item \texttt{theoremsetup.tex} declares the theorem-like environments.
\item \texttt{macrosetup.tex} defines extra macros that you may find
  useful.
\item \texttt{introduction.tex} contains this text.
\item \texttt{sections.tex} is a quick demo of each sectioning level
  available.
\item \texttt{refs.bib} is an example bibliography file.  You can use
  Bib\TeX{} to quote references.  For example, read
  \cite{bringhurst1996ets} if you can get a hold of it.
\end{enumerate}


\subsection{Extra package includes}

The file \texttt{extrapackages.tex} lists some packages that usually
come in handy.  Simply have a look at the source code.  We have
added the following comments based on our experiences:
\begin{description}
\item[REC] This package is recommended.
\item[OPT] This package is optional.  It usually solves a specific
  problem in a clever way.
\item[ADV] This package is for the advanced user, but solves a problem
  frequent enough that we mention it. Consult the package's
  documentation.
\end{description}

As a small example, here is a reference to the Section \emph{Features}
typeset with the recommended \package{varioref} package:
\begin{quote}
  See Section~\vref{sec:features}.
\end{quote}


\subsection{Layout setup}

This defines the overall look of the document -- for example, it changes the chapter and section heading appearance.  We consider this a `do not touch' area.  Take a look at the excellent \emph{Memoir} documentation before changing it.

In fact, take a look at the excellent \emph{Memoir} documentation, full stop.