% Some commands used in this file
\newcommand{\package}{\emph}


\chapter{Introduction}

\section{Introduction}
\subsection{Motivation and Aim}
\label{sec:motivation}
Mechanosensation describes the phenomenon of cells sensing external physical stimuli. \myworries{Add an example, like aligning cells in groove\dots} This modality adds to the pre-existing paradigm of biological and chemical influences that cells can sense and react to. Mechanosensation describes one aspect of mechanobiology, which describes next to the physical stimuli, cells react to, the reaction of a cell to external physical stimuli on every level (i.e. from sensing to changing gene expression profiles).  With new publications almost weekly and new journals emerging, it becomes clear that mechanobiology is of far reaching consequences and very complex, with a multitude of biological sciences affected from ground up. 
\par 

Unquestionably, mechanobiology promises great advances in the field of regenerative medicine. This relatively new field of biomedical research is driven by the vision to restore the physiological status in a patient, by means of supporting endogenic restorative capabilities and/or implantation of biogenic replacement often with cellular components that help to ameliorate the condition. \myworries{Give an example of a promising project in a regenerative medicine.} Even though the intentions are clear, there are still many open questions, that hinder systematic success. Therefore, it is imperative to gain a thorough understanding of the target system and its development\myworries{emergence, embryological origin}. For this reason, research in regenerative medicine is heavily implicated with stem cells. Adding to this reason, stem cells, by their nature, have reproductive and differentiating potential. This versatile quality render them pivotal to therapies emerging from regenerative medicine. \par

\subsection{Hypothesis and Goal}


\section{Theory}

\subsection{Piezo1}
Described in 200X for the first time, Piezo1 is a  mechanically activated ion channel abundantly expressed in all eukaryotes including plants. ???\cite{Coste55} \\
Three different states (active, inactive and non-reactive)\\
Blade Structure, Multipass Protein, Effector unit and sensor unit, highly conserved across species\\
Peculiar structure and evolutionary properties, next to abundance and recent finding, make it a very interesting object of study. 

\subsection{Yoda1}
Piezo1 selective Agonist. Binds on outer leaflet, transmits force according to
As we shall see in~\vref{sec:biostability} Also is unstable in physiological conditions of \myworries{37 degrees Celsius}

`Our results suggest a model wherein Yoda1 acts as a molecular wedge, facilitating force-induced conformational changes, effectively lowering the channel's mechanical threshold for activation.'\cite{BotelloSmith.2019} 

\subsection{Mesenchymal Stem Cells}
The CDXX++/CDXX--/CDXX++ mesenchymal stem cells are produced during embryogenesis, they go through\dots Until they finally become the pre-cells to\dots 
While there are different types of cells that could be understood to be mesenchymal stem cells, with differences regarding potency and hematopoietic potential, in this work we will focus entirely on human bone-marrow derived mesenchymal stem cells. 

\section{Extra package includes \myworries{TOBEDELETED!}}

The file \texttt{extrapackages.tex} lists some packages that usually
come in handy.  Simply have a look at the source code.  We have
added the following comments based on our experiences:
\begin{description}
\item[REC] This package is recommended.
\item[OPT] This package is optional.  It usually solves a specific
  problem in a clever way.
\item[ADV] This package is for the advanced user, but solves a problem
  frequent enough that we mention it. Consult the package's
  documentation.
\end{description}

As a small example, here is a reference to the Section \emph{Motivation}
typeset with the recommended \package{varioref} package:
\begin{quote}
  See Section~\vref{sec:motivation}.
\end{quote}
