
\chapter{Introduction}

\section{Introduction}
\subsection{Motivation}
\label{sec:motivation}
Mesenchymal stem cells will disrupt the medicine of tomorrow and will become a routine biomedical intervention in the rising world of personalized medicine. 
Mesenchymal stem cells (MSC) were described the first time in CHECK IN MSC REVIEW PAPER. 

MSC can give rise to various daughter cells found in fat- (adipocytes), cartilage- (chondrocytes), bone- (osteoblasts) and tendon-tissue (tenocytes) \cite{Barlow2008, Hass2011} and as such are important development.
They persist during adult life, and can be found in NAME DIFFERENT TISSUES, while being easily isolated and cultured. Because of this versatility, their pronounced capacity for self-renewal and their robustness, plenty of research is being done to exploit this potential and provide medical application. Baek and colleagues recently described a way of ex vivo manipulating the MSCs homing behaviour to stimulate migration towards injured tissue, before injecting them back in the donor, which could in the future improve recovery from myocardial infarction. \cite{Baek2011}  Additional proposals for novel applications include the fields of spinal injuries\cite{Goldschlager2010}, Multiple Sclerosis \cite{Planchon2018}, Alzheimer's Disease \cite{Han2018, Hao2012} and Diabetes \cite{Evangelista2018}.  While the promises that MSCs hold might be worthwhile, success in clinics is missing. Despite the prospect of a medical revolution, currently there is no FDA approved MSC therapy available, as past proposals have been shown to either ineffective or dangerous.\cite{Amariglio2009}. This gap between theoretical possibility and practical feasibility highlights the urgent need for basic MSC research.\par

The historic paradigm in cell biology that a system is described only through its chemical status, has come into question during the last years. We now know that mechanical forces which cells experience \textit{in vivo} influence their behaviour significantly\cite{Hao2015}. Mechanosensing, the ability to “feel forces”, is a contributing factor in almost all cellular contexts from migration over tissue effector function to differentiation and self-renewal. In a landmark study, Engler and colleagues\cite{Engler2006} showed that MSCs can decide on their cell fate only due to physical cues like substrate stiffness. Next to a remarkable discovery by itself, this study is one of many more following accounts showing that there is new, decisive knowledge in studying biophysical interaction between cells and their surroundings.\par

However, it is not fully understood what pathways are involved and how they relate to each other.  More recently, the membrane-bound mechanosensitive ion-channel \Piezo{} has been described \cite{Coste2010}. The discovery marked a turning point in mechanobiology as it described the sought-after description of the first eukaryotic mechanosensitive ion channel. Since then, many studies have corroborated the idea of \Piezo{} as belonging to a family of receptors who have already been predicted to exist since 1950 \cite{Katz1949}. While we now know a lot about how \Piezo{} is involved in THERE and THERE, it is not fully understood how \Piezo{} influences MSC behaviour. \par
In this work, we investigate the importance of \Piezo{} in MSC mechanosensing and shed light on its influence regarding ECM turnover and differentiation.


DRAFT: 


The existence of MechanICs was already postulated in 1950 in a study measuring a change of electric potential in response to stretching of muscle \cite{Katz1949}. Over the years, the existence of MechanICs became increasingly clear, as studies showed them to contribute to a variety of physiological functions, like regulating the myogenic tone of resistance arteries in muscle \cite{Murthy2017} or stimulation of vasodilatation in the vascular system \cite{Zeng2018}. Especially, in pain research, were neurons implicated in pain sensation seem to rely on MechanICs, the necessity for discovery was urgent. Notwithstanding the relevance, the discovery of the first gene encoding a MechanIC would take another 44 years, until Sukharev and colleagues describe two distinct genes mscL and mscS, whose product mediate stretch sensing in the membrane of E. Coli bacteria. While the efforts and papers, that were inspired by this discovery, gave great insight into structure and biophysical mechanism of those channels, eukaryotic homologues were still missing. Until recently, the only eukaryotic candidate was TREK-1, a potassium-selective ion channel, whose mechanosensitive property is still controversial. 

Tendon is collagenous and highly tensile tissue that connects muscle to bone, acting as a force transmission and effectively enabling us to move. 

Add Comment about tendon metabolism. 

The well-being of our tendon is crucial, as patients with tendon-related diseases suffer a lot and they are increasing in number with a projected amount of XX Patients in 20XX. The most prevalent tendon related disease next to rupture is called tendinopathy and describes an inflammatory disease of the tendon, with most common symptoms reported being movement associated pain, swelling and substantial decrease in quality of life. On a clinical level erroneous blood vessel formation inside the normally non-vascularized tendon core (vascular in-growth) is the hallmark of this disease. Currently, the pathophysiology is poorly understood. This lack of understanding regarding tendinopathy etiology leads to the treatment being largely symptomatic. 

Some cases progress and make surgical intervention necessary. In those cases, the surgeon can cut out the diseased part, hence the need for tendon replacement surgery.  One approach is to transfer tendon from a non-diseased site. *Explain Problems that we are left with and how mesenchymal stem cells can help us* They hold great promises, but systematic success is still lacking. This is largely due to undefined behaviour. *Explain reasons for why they do not work now* 

More and more, it becomes clear that we need to understand the dynamics better that underlie MSCs before we can start utilizing their full potential.\par 

Mechanobiology is this domain of biology at the interface between cells and their physical environment. 
Even though the first hypothesis that involve cells being able to sense physical stimuli dates back to the late 19th century, the pathways implicated in mechanobiology are heavily researched and mechanobiology is rather recently gaining a lot of scientific momentum. Eyckman and colleagues suggest a lack in technology as main reason for this delay from conceptualisation to investigation \cite{Eyckmans2011}. 
The identification of mechanically gated channel (MechanIC) named \Piezo{} was very important to push the frontier of the field. \par
\Piezo{} is a calcium-permeable non-selective ion-channel located at the membrane. Its discovery in 2009 marks a groundbreaking advancement for mechanobiology and cell biology in general as it is the first description of a mechanosensitive ion channel found in eukaryotic cells.\cite{Coste2010} 
Finally, in their breakthrough paper from 2010, Coste and colleagues report the discovery of two genes Fam38A and Fam38B, which encode mechanosensitive protein Piezo1 and Piezo2, respectively. Even though they carefully avoided calling Piezo1 an ion channel in their first paper, a follow-up study in the same lab confirmed piezo proteins as subunits for the homo-trimeric MechanIC Piezo1. Finally, scientists characterized the first eukaryotic mechanosensitive ion channel. 


\subsection{Aim}


\section{Hypothesis}

\begin{itemize}
	\item Sass
	\item 
\end{itemize}


\section{Theory}

\subsection{\Piezo{}}

\myworries{ADD NICE PIEZO1-Picture!}


\Piezo{} and \textsc{Piezo2} are the only known homologues belonging to the \textsc{Piezo}-family. 
The protein family is genetically and structurally unique, as the similarity to other proteins are minimal \cite{Coste2010}. 
The ubiquitously expressed protein \Piezo{} constitutes the mechanosensitive homo-trimeric cation channel \Piezo{}. \cite{Zhao2018} This multi-transmembrane channel has a unique molecular architecture as shown in an extraordinarily elegant paper by Saotome and colleagues giving great insight into \Piezo{}-structure using cryo-EM. The large channel comprises a central pore in the middle with a three-bladed propeller like structure facing the extracellular space (see Fig) \cite{Saotome2018}. When the cellular membrane is exposed to a mechanical force exceeding a certain threshold, \Piezo{} switches into an open state where it allows calcium and other cations to enter the intracellular space, before switching into an inactive state, where the channel is non-responsive to stimuli. After relaxation period it switches to a responsive, closed state. This concludes the 3-state channel model, which describes \Piezo{}. While the exact activation mechanism remains elusive, there is evidence for the mechanotransductory chain starting with force-mediated bending of blade region which then transmits the tension over the anchor/beam region, leading to the opening of a previously allosterically closed inner portion of the cation channel \cite{Zhao2018}.
Alternatively, there is a chemical \Piezo{}-agonist, called \Yoda{} (a nod at movie director George Lucas’ use of “the Force”). \Yoda{} has been shown to chemically activate \Piezo{} by acting as a molecular wedge. \cite{Syeda2015, Lacroix2018}. This can be used to study the influence of \Piezo{} in an isolated design  \cite{Botello-Smith2019}.

\subsection{Bone Marrow-derived Mesenchymal Stem Cells}

check Bronw et al2019, Mesenchymal stem cells: Cell therapy and regeneration potential (https://onlinelibrary.wiley.com/doi/full/10.1002/term.2914). 

MSC's are characterized by four different properties: First, they hold the potential to differentiate into a defined set of daughter cells, including osteocyte (bone), chondrocyte (cartilage), adipocyte (fat-tissue) and tenocyte (tendon).\cite{Ng2008} Second, they hold the potential to self-renew, meaning they can give rise to identical cells. Third, they expose a set of stem cell markers like CD31, CD34 and cKit. While those markers are associated with stem cells, it is important to keep in mind that they are not unique to them. \cite{Battula2009} And fourth, they can be conveniently isolated from primary tissue based on their capacity to stick on plastic. \cite{Buhring2007}. Furthermore, there is a growing body of evidence that attributes an immunomodulatory effect of MSC on injured or inflamed sites, which could relate to better outcome in medical intervention \cite{Hass2011, Caplan2011}.
While there are differing anatomical sources where mesenchymal stem cells can be isolated from, including the umbilical cord and adipose tissue \cite{Barlow2008, Hass2011} in this work we are going to focus on human bone-marrow derived mesenchymal stem cells and from now on will refer to them as mesenchymal stem cells (MSCs).

\subsection{Shear force mechanosensing}
Definition of shear forces. Locate them in different physical forces in the body (stretch, compression,) Distinguish also from passive physical properties, that are then explored by the cell (e.g. substrate stiffness, topology).
Physiological context (different shear: fluidic vs condensed)
Although, in literature the name \Piezo{} is used interchangeably for the channel itself as also the protein that constitutes the channel, in this work we will from now one refer to the channel as Piezo1.

