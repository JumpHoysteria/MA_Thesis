% Some commands used in this file
\newcommand{\package}{\emph}


\chapter{Introduction}

\section{Introduction}
\subsection{Motivation and Aim}
\label{sec:motivation}
To be deleted 
\begin{itemize}
    \item Regenerative medicine is a relatively new field of biomedical research, which is driven by the vision to restore the physiological status in a patient, by means of supporting endogenic restorative capabilities and/or implantation of biogenic substitutes often with cellular components that help to ameliorate the condition.The average person is always getting older and age-associated disease will become more prevalent. This increases pressure for medical therapies that address them. The world wide market for regenerative market is said to grow 20\% each year, and is projected to reach around USD 73 billion by the year 2027.[MarketsAndMarkets, https://cutt.ly/WrUSspf] The growing tendency can be explained because of the paradigm of regenerative therapy, which is restoration and reconstitution of patient. Some examples are cardiac repair [Pittenger2004] and applications in spine-surgery [Goldschlager2010]. However, there is one that deserves special mention: Mesenchymal Stem cell applications to tendon healing. Even though the goal is clear, there are still many open questions, that hinder systematic success. Since XXXX, there are studies being done were cells are co-transplanted to ameliorate healing and thought to eventually lead to a better patient outcome. This strategy is derived from the school of thought, that cells can direct other cells behaviour. Either by direct interaction (e.g. by secretion of cytokines or para-hormones) or through secondary pathways like orchestrating ECM-composition that in turn helps in the coordination of cellular behaviour and implant biointegration. Stem cells are often the preferred choice, since they have the property of self-renewal and a lot of differentiation potential. 
\myworries{Immunomodulatory and relative abundance?} This versatile qualities render them pivotal to therapies emerging from regenerative medicine. Still a lot of open questions of the inner workings from MSC's. One of the questions is what channels are involved in physical force sensing in MSCs and how they react to it. 
\end{itemize}


\section{Theory}

\subsection{\Piezo{}}

\myworries{ADD NICE PIEZO1-Picture!}

\Piezo{} and \textsc{Piezo2} are the only members of the \textsc{Piezo}-family. The protein family is genetically and structurally unique, as the similarity to other proteins are minimal [Coste2008]. In this work we only focus on \Piezo{}, which constitutes the mechanosensitive homo-trimeric ion channel \Piezo{}, which has a unique architecture resembling a propeller with three blades, where each of those blades are constituted by one \Piezo{} Protein. Functionally, the channel can be described the central pore, the anchor and beam region and the outer leaflets or beams. 

\begin{itemize}
	\item different views on count of transmembrane regions Saotome2018
	\item Anchor and Beam Connect Blade to conducting pore Saotome2018
	\item Hyphothesis that actually force transduction from blade to pore is opening mechanism, but the exact biophysical mechanism remains elusive. 
\end{itemize}

 When in an open state, it selectively permits passage of extracellular calcium, which can then acts as a cofactor further downstream. [Coste2010]. The channel is expressed almost ubiquitously in our body. Furthermore, \Piezo{} is evolutionary preserved with analogues spanning multiple evolutionary kingdoms including plants. [Coste2010][Coste2012]\\
Although, in literature the name \Piezo{} is used interchangeably for the channel itself as also the protein that constitutes the channel, in this work we will from now one refer to the channel as \Piezo{}.\\
\Yoda{} is a small molecule, able to chemically activate \Piezo{} selectively. [Syeda2015] The proposed mechanism is that Yoda1 acts as a molecular wedge, lowering the mechanical threshold for activation. [Botello-Smith2019] This allows Yoda1 to partially activate \Piezo{} in absence of externally applied forces. [Lacroix2018]

\subsection{Bone Marrow-derived Mesenchymal Stem Cells}
MSC's are characterized by four different properties: First, they hold the potential to differentiate into a defined set of daughter cells, including osteocyte (bone), chondrocyte (cartilage), adipocyte (fat-tissue) and tenocyte (tendon).[Ng2008] Second, they hold the potential to self-renew, meaning they can give rise to identical cells. Third, they expose a set of stem cell markers like CD31, CD34 and cKit. While those markers are associated with stem cells, it is important to keep in mind that they are not unique to them. [Battula2009] And fourth, they can be conveniently isolated from primary tissue based on their capacity to stick on plastic. [Bühring2007] All of the beforementioned properties Furthermore, there is a growing body of evidence that attributes an immunomodulatory effect of MSC on injured or inflamed sites, which could relate to better outcome in medical intervention. [Hass2011], [Caplan2011]
While there are differing anatomical sources where mesenchymal stem cells can be isolated from, including the umbilical cord and adipose tissue [Barlow2008], [Hass 2011] in this work we are going to focus on human bone-marrow derived mesenchymal stem cells and from now on will refer to those cells as mesenchymal stem cells (MSCs).


\subsection{Shear force mechanosensing}
Definition of shear forces. Locate them in different physical forces in the body (stretch, compression,) Distinguish also from passive physical properties, that are then explored by the cell (e.g. substrate stiffness, topology).
Physiological context (different shear: fluidic vs condensed)
